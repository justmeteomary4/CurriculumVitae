%%%%%%%%%%%%%%%%%%%%%%%%%%%%%%%%%%%%%%%%%%%%%%%%%%%%%%%%%%%%%%%%%%%%%%%
%% NOTE: If you find that it says                                     %%
%%                                                                    %%
%%                           1 of ??                                  %%
%%                                                                    %%
%% at the bottom of your first page, this means that the AUX file     %%
%% was not available when you ran LaTeX on this source. Simply RERUN  %%
%% LaTeX to get the ``??'' replaced with the number of the last page  %%
%% of the document. The AUX file will be generated on the first run   %%
%% of LaTeX and used on the second run to fill in all of the          %%
%% references.                                                        %%
%%%%%%%%%%%%%%%%%%%%%%%%%%%%%%%%%%%%%%%%%%%%%%%%%%%%%%%%%%%%%%%%%%%%%%%%

\documentclass[10pt]{article}
%\usepackage{fontawesome} % E-mail, Twitter logos

% The automated optical recognition software used to digitize resume
% information works best with fonts that do not have serifs. This
% command uses a sans serif font throughout. Uncomment both lines (or at
% least the second) to restore a Roman font (i.e., a font with serifs).
%\usepackage{times}
%\renewcommand{\familydefault}{\sfdefault}
%\setmainfont{}

% This is a helpful package that puts math inside length specifications
\usepackage{calc}
\usepackage{comment}
\usepackage{amssymb}
\usepackage{MnSymbol}

\usepackage{datetime}

% Simpler bibsection for CV sections (thanks to natbib for inspiration)
\makeatletter
\newlength{\bibhang}
\setlength{\bibhang}{1em} %1em}
\newlength{\bibsep}
 {\@listi \global\bibsep\itemsep \global\advance\bibsep by\parsep}
\newenvironment{bibsection}%
        {\begin{enumerate}{}{%
%        {\begin{list}{}{%
       \setlength{\leftmargin}{\bibhang}%
       \setlength{\itemindent}{-\leftmargin}%
       \setlength{\itemsep}{\bibsep}%
       \setlength{\parsep}{\z@}%
        \setlength{\partopsep}{0pt}%
        \setlength{\topsep}{0pt}}}
        {\end{enumerate}\vspace{-.6\baselineskip}}
%        {\end{list}\vspace{-.6\baselineskip}}
\makeatother

% Layout: Puts the section titles on left side of page
\reversemarginpar

%
%         PAPER SIZE, PAGE NUMBER, AND DOCUMENT LAYOUT NOTES:
%
% The next \usepackage line changes the layout for CV style section
% headings as marginal notes. It also sets up the paper size as either
% letter or A4. By default, letter was used. If A4 paper is desired,
% comment out the letterpaper lines and uncomment the a4paper lines.
%
% As you can see, the margin widths and section title widths can be
% easily adjusted.
%
% ALSO: Notice that the includefoot option can be commented OUT in order
% to put the PAGE NUMBER *IN* the bottom margin. This will make the
% effective text area larger.
%
% IF YOU WISH TO REMOVE THE ``of LASTPAGE'' next to each page number,
% see the note about the +LP and -LP lines below. Comment out the +LP
% and uncomment the -LP.
%
% IF YOU WISH TO REMOVE PAGE NUMBERS, be sure that the includefoot line
% is uncommented and ALSO uncomment the \pagestyle{empty} a few lines
% below.
%

%% Use these lines for letter-sized paper
%\usepackage[paper=letterpaper,
%            %includefoot, % Uncomment to put page number above margin
%            marginparwidth=1.2in,     % Length of section titles
%            marginparsep=.05in,       % Space between titles and text
%            margin=1in,               % 1 inch margins
%            includemp]{geometry}

%% Use these lines for A4-sized paper
\usepackage[paper=a4paper,
            includefoot, % Uncomment to put page number above margin
            marginparwidth=42mm,      % Length of section titles 30.5mm
            marginparsep=1.5mm,       % Space between titles and text
            margin=12mm,              % 25mm margins
            includemp]{geometry}

%% More layout: Get rid of indenting throughout entire document
\setlength{\parindent}{0in}

\usepackage[shortlabels]{enumitem}

%% Reference the last page in the page number
%
% NOTE: comment the +LP line and uncomment the -LP line to have page
%       numbers without the ``of ##'' last page reference)
%
% NOTE: uncomment the \pagestyle{empty} line to get rid of all page
%       numbers (make sure includefoot is commented out above)
%
\usepackage{fancyhdr,lastpage}
\pagestyle{fancy}
\pagestyle{empty}      % Uncomment this to get rid of page numbers
\fancyhf{}\renewcommand{\headrulewidth}{0pt}
\fancyfootoffset{\marginparsep+\marginparwidth}
\newlength{\footpageshift}
\setlength{\footpageshift}
          {0.5\textwidth+0.5\marginparsep+0.5\marginparwidth-2in}
\lfoot{\hspace{\footpageshift}%
       \parbox{4in}{\, \hfill %
                    \arabic{page} of \protect\pageref*{LastPage} % +LP
%                    \arabic{page}                               % -LP
                    \hfill \,}}

% Finally, give us PDF bookmarks
\usepackage{color,hyperref}
\definecolor{darkblue}{rgb}{0.0,0.0,0.25}
\hypersetup{colorlinks,breaklinks,
            linkcolor=darkblue,urlcolor=darkblue,
            anchorcolor=darkblue,citecolor=darkblue}

\usepackage{graphicx}
\usepackage{wrapfig}
%%%%%%%%%%%%%%%%%%%%%%%% End Document Setup %%%%%%%%%%%%%%%%%%%%%%%%%%%%


%%%%%%%%%%%%%%%%%%%%%%%%%%% Helper Commands %%%%%%%%%%%%%%%%%%%%%%%%%%%%

% The title (name) with a horizontal rule under it
% (optional argument typesets an object right-justified across from name
%  as well)
%
% Usage: \makeheading{name}
%        OR
%        \makeheading[right_object]{name}
%
% Place at top of document. It should be the first thing.
% If ``right_object'' is provided in the square-braced optional
% argument, it will be right justified on the same line as ``name'' at
% the top of the CV. For example:
%
%       \makeheading[\emph{Curriculum vitae}]{Your Name}
%
% will put an emphasized ``Curriculum vitae'' at the top of the document
% as a title. Likewise, a picture could be included:
%
%   \makeheading[\includegraphics[height=1.5in]{my_picutre}]{Your Name}
%
% the picture will be flush right across from the name.
\newcommand{\makeheading}[2][]%
        {\hspace*{-\marginparsep minus \marginparwidth}%
         \begin{minipage}[t]{\textwidth+\marginparwidth+\marginparsep}%
             {\large \bfseries #2 \hfill #1}\\[-0.15\baselineskip]%
                 \rule{\columnwidth}{1pt}%
         \end{minipage}}

% The section headings
%
% Usage: \section{section name}
\renewcommand{\section}[1]{\pagebreak[3]%
    \hyphenpenalty=10000%
    \vspace{1.3\baselineskip}%
    \phantomsection\addcontentsline{toc}{section}{#1}%
    \noindent\llap{\scshape\smash{\parbox[t]{\marginparwidth}{\raggedright #1}}}%
    \vspace{-\baselineskip}\par}

% An itemize-style list with lots of space between items
\newenvironment{outerlist}[1][\enskip\textbullet]%
        {\begin{itemize}[#1,rightmargin=2mm,leftmargin=*]}{\end{itemize}%
         \vspace{-.6\baselineskip}}

% An environment IDENTICAL to outerlist that has better pre-list spacing
% when used as the first thing in a \section
\newenvironment{lonelist}[1][\enskip\textbullet]%
        {\begin{list}{#1}{%
        \setlength{\partopsep}{0pt}%
        \setlength{\topsep}{0pt}}}
        {\end{list}\vspace{-.6\baselineskip}}

% An itemize-style list with little space between items
\newenvironment{innerlist}[1][\enskip$\filledsquare$]%
        {\begin{itemize}[#1,leftmargin=*,rightmargin=1mm,parsep=0pt,itemsep=0pt,topsep=0pt,partopsep=0pt]}
        {\end{itemize}}
        
\newenvironment{list2}[1][\enskip$\filledsquare$]%
        {\begin{itemize}[#1,leftmargin=*,rightmargin=4cm,parsep=0pt,itemsep=0pt,topsep=0pt,partopsep=0pt]}
        {\end{itemize}}

% An environment IDENTICAL to innerlist that has better pre-list spacing
% when used as the first thing in a \section
\newenvironment{loneinnerlist}[1][\enskip\textbullet]%
        {\begin{itemize}[#1,leftmargin=*,parsep=0pt,itemsep=0pt,topsep=0pt,partopsep=0pt]}
        {\end{itemize}\vspace{-.6\baselineskip}}

% To add some paragraph space between lines.
% This also tells LaTeX to preferably break a page on one of these gaps
% if there is a needed pagebreak nearby.
\newcommand{\blankline}{\quad\pagebreak[3]}
\newcommand{\halfblankline}{\quad\vspace{-0.5\baselineskip}\pagebreak[3]}

% Uses hyperref to link DOI
\newcommand\doilink[1]{\href{http://dx.doi.org/#1}{#1}}
\newcommand\doi[1]{doi:\doilink{#1}}

% For \url{SOME_URL}, links SOME_URL to the url SOME_URL
\providecommand*\url[1]{\href{#1}{#1}}
% Same as above, but pretty-prints SOME_URL in teletype fixed-width font
\renewcommand*\url[1]{\href{#1}{\texttt{#1}}}

% For \email{ADDRESS}, links ADDRESS to the url mailto:ADDRESS
\providecommand*\email[1]{\href{mailto:#1}{#1}}
% Same as above, but pretty-prints ADDRESS in teletype fixed-width font
%\renewcommand*\email[1]{\href{mailto:#1}{\texttt{#1}}}

%\providecommand\BibTeX{{\rm B\kern-.05em{\sc i\kern-.025em b}\kern-.08em
%    T\kern-.1667em\lower.7ex\hbox{E}\kern-.125emX}}
%\providecommand\BibTeX{{\rm B\kern-.05em{\sc i\kern-.025em b}\kern-.08em
%    \TeX}}
\providecommand\BibTeX{{B\kern-.05em{\sc i\kern-.025em b}\kern-.08em
    \TeX}}
\providecommand\Matlab{\textsc{Matlab}}

%\let\mathdollar\undefined % uncomment with using fontawesome
%%%%%%%%%%%%%%%%%%%%%%%% End Helper Commands %%%%%%%%%%%%%%%%%%%%%%%%%%%

%%%%%%%%%%%%%%%%%%%%%%%%% Begin CV Document %%%%%%%%%%%%%%%%%%%%%%%%%%%%

\begin{document}

% Timestamp, comment if want to show photo instead
\flushright{\small \texttt{\shortmonthname[\the\month] \the\year}}\flushleft

\makeheading[CV]{Maria Zamyatina}

% Uncomment to show photo
%\begin{figure}
%\includegraphics[width=0.25\textwidth]{photoCV_whitebackground.jpg}
%\end{figure}

\section{Personal Information}

% NOTE: Mind where the & separators and \\ breaks are in the following
%       table.
%
% ALSO: \rcollength is the width of the right column of the table
%       (adjust it to your liking; default is 1.85in).
%
%\newlength{\rcollength}\setlength{\rcollength}{1.4in}%
%
%\begin{tabular}[t]{@{}p{\textwidth-\rcollength}p{\rcollength}}
%\end{tabular}%
%University of East Anglia, Norwich Research Park, Norwich, NR4 7TJ, UK\\
%\faEnvelope~m.zamyatina@uea.ac.uk \faTwitter~chemeteomary \faGithub~chemarista \\
\includegraphics[scale=0.02,decodearray={.1 .5 .5 .5 .5 .5}]{email_logo_fontawesome.png} m.zamyatina@uea.ac.uk 
\includegraphics[scale=0.02,decodearray={.1 .5 .5 .5 .5 .5}]{twitter_logo_fontawesome.png} \href{https://twitter.com/m_zamyatina}{m$\_$zamyatina}
\includegraphics[scale=0.02,decodearray={.1 .5 .5 .5 .5 .5}]{github_logo_fontawesome.png} \href{https://github.com/mzamyatina}{mzamyatina} \\
\includegraphics[scale=0.02,decodearray={.1 .5 .5 .5 .5 .5}]{website_logo_fontawesome.png} \href{https://mariazamyatina.wordpress.com/}{mariazamyatina.wordpress.com}

\section{Research Interests}
atmospheric chemistry, numerical modelling, climate change

\section{Education}
\textbf{PhD in Atmospheric Chemistry}  \hfill {2015-expected 2019} \\
\href{https://www.uea.ac.uk/environmental-sciences/}{School of Environmental Sciences, University of East Anglia} $|$ Norwich, UK \\
Supervisors: \href{https://people.uea.ac.uk/c_reeves}{Prof. Claire Reeves}, \href{https://www.ecmwf.int/en/about/who-we-are/staff-profiles/marcus-koehler}{Dr. Marcus K\"ohler}, \href{https://www.york.ac.uk/chemistry/staff/resstaff/mnewland/}{Dr. Mike Newland} \\
Thesis: Impact of alkyl nitrate chemistry on tropospheric ozone: box and global model \\
\hspace{1.1cm} perspectives
\begin{innerlist}
\item updated chemical kinetics of \href{http://www.ukca.ac.uk/wiki/index.php/UKCA}{UKCA}'s standard tropospheric chemistry mechanism
\item added new alkyl nitrate chemistry into UKCA
\item tested new mechanism against the \href{http://mcm.leeds.ac.uk/MCM/home.htt}{Master Chemical Mechanism (MCM)} in a box model 
\item validated UKCA with the \href{https://espo.nasa.gov/atom}{Atmospheric Tomography Mission (ATom)} aircraft data
\item analysed $\text{HO}_\text{x}$, $\text{NO}_\text{x}$, $\text{NO}_\text{y}$ burdens and distribution in UKCA long-term runs with/without alkyl nitrate photochemical production/direct emissions
\item contributed to the \href{https://www.researchgate.net/project/Oxidant-Budgets-of-the-Northern-Hemisphere-Troposphere-Since-1950-OXBUDS}{Oxidant Budgets of the Northern Hemisphere Troposphere Since 1950 (OXBUDS)} project
\end{innerlist}

\vspace{.1in}
\textbf{MSc in Climate Change with Distinction}  \hfill {2014-2015} \\
School of Environmental Sciences, University of East Anglia $|$ Norwich, UK \\
Supervisor: Prof. Claire Reeves \\
Thesis: Investigation of the relationship between tropospheric ozone production efficiency \\ \hspace{1.1cm} and carbon bond emissions

\vspace{.1in}
\textbf{Specialist Diploma in Meteorology}  \hfill {2009-2014} \\
\href{http://www.eng.geogr.msu.ru/}{Faculty of Geography, Lomonosov Moscow State University} $|$ Moscow, Russia \\
Supervisor: \href{https://www.researchgate.net/profile/Alexander_Kislov2}{Prof. Alexander V. Kislov} \\
Thesis: Climatically-induced variations of the Caspian Sea level over the last Millennium

%\section{Conferences}
\section{Talks}
\begin{innerlist}
\item[Apr 2019] \href{https://meetingorganizer.copernicus.org/EGU2019/EGU2019-11151.pdf}{Impact of $\text{C}_\text{1}$-$\text{C}_\text{3}$ alkyl nitrate chemistry on tropospheric ozone: box and
global model perspectives}
\item[] EGU $|$ Vienna, Austria
%{\bf Zamyatina M.}, Reeves C.E., Archibald A.T., Griffiths P.T., K\"ohler M.O.
%Geophysical Research Abstracts, Vol. 21, EGU2019-11151, 2019.
%7-12 April 2019
%\item[Mar 2019] Impact of $\text{C}_\text{1}$-$\text{C}_\text{3}$ alkyl nitrate chemistry on tropospheric ozone: box and global model perspectives
%\item[] \href{https://www.uea.ac.uk/environmental-sciences/news-and-events/geochemical-luncheon-club-series/glc-past}{Atmospheric and marine biogeochemistry seminar} $|$ University of East Anglia, UK
%{\bf Zamyatina M.}, Reeves C.E., Archibald A.T., Griffiths P.T., K\"ohler M.O.
%18 March 2019
\item[Apr 2017] Adding new chemistry into UM-UKCA
\item[] Cambridge-EnvEast Doctoral Alliance Symposium $|$ Cambridge, UK
%{\bf Zamyatina M.}, Reeves C.E., Archibald A.T., Griffiths P.T., Newland M.J., K\"ohler M.O
%11 April 2017.
\item[Sep 2012] Assessment of climatological potential of transboundary air pollution transport in Eastern Siberia and the Russian Far East
\item[] \href{http://siga.uubf.itu.edu.tr/atmosfer/index.php/aqm/aqm2012/index}{Air Quality Management at Urban, Regional and Global Scales 4th International Symposium/IUAPPA Regional Conference} $|$ Istanbul, Turkey
%Gromov S.A., Gromov S.S., {\bf Zamyatina M.}.
%10-13 September 2012.
\end{innerlist}

\section{Posters}
\begin{innerlist}
\item[Sep 2018] Impact of alkyl nitrate chemistry on tropospheric ozone 
\item[] \href{http://icacgp-igac2018.org/}{2018 joint 14th iCACGP  Quadrennial Symposium/15th IGAC  Science Conference} $|$ Takamatsu, Japan
%{\bf Zamyatina M.}, Reeves C.E., Archibald A.T., Griffiths P.T., K\"ohler M.O.
%Poster 2.092
%25-29 September 2018
\item[Apr 2018] \href{https://meetingorganizer.copernicus.org/EGU2018/EGU2018-5206.pdf}{Impact of $\text{C}_\text{1}$-$\text{C}_\text{5}$ alkyl nitrate chemistry on tropospheric ozone - a box modelling study} 
\item[] EGU $|$ Vienna, Austria
%\bf Zamyatina M.}, Reeves C.E., Archibald A.T., Griffiths P.T., K\"ohler M.O.
%Geophysical Research Abstracts, Vol. 20, EGU2018-5206, 2018.
%8-13 April 2018
%\item[Mar 2018] Impact of $\text{C}_\text{1}$-$\text{C}_\text{5}$ alkyl nitrate chemistry on tropospheric ozone - a box modelling study
%\item[] Cambridge-EnvEast Doctoral Alliance Symposium $|$ Cambridge, UK
%19-20 March 2018.
\end{innerlist}

\section{Proceedings}
%\begin{bibsection}
\begin{innerlist}
\item Gromov S.A., Gromov S.S., {\bf Zamyatina M.}, Trifonova-Yakovleva A.M. 2013. First-order evaluation of transboundary pollution fluxes in areas of EANET stations in Eastern Siberia and the Russian Far East. \href{https://www.eanet.asia/wp-content/uploads/2019/04/SciBull_Sep_2013.pdf}{EANET Science Bulletin, 3}:195-203.
%\item Sergeev D., {\bf Zamyatina M.}, Stepanenko V. 2013. \emph{Thermal regime features of Kronotsky lake (Kronotsky Nature Reserve)}. \href{http://istina.msu.ru/publications/article/5336939/}{Kronotsky State Natural Biosphere Reserve Proceedings, Vol. 3. p. 29-41, 2013} (in Russian).
\item Sergeev D., {\bf Zamyatina M.}, Stepanenko V. 2013. \href{http://istina.msu.ru/publications/article/5336939/}{Thermal regime features of Kronotsky lake} (in Russian). Kronotsky State Natural Biosphere Reserve Proceedings, 3: 29-41.
%\item Barabanova O.V., Fedorov G.A., Khrupolova E.A., Konstantinov P.I., Kukanova E.A., Malinina E.P., Sergeev D.E., Sokolova L.A., Stepanenko V.M., Varentsov M.V., Veresemskaya P.S., {\bf Zamyatina M.Yu.}, Zheleznova I.V. \emph {Experimental investigation and remote sensing of boundary layer in high latitudes (evidence from the coastal zone of the White Sea)}. \href{http://lomonosov-msu.ru/archive/Lomonosov_2012/structure_6_1716.htm}{Proceedings of the International Youth Science Forum Lomonosov-2012}, Moscow, Russia, 2012 (in Russian).
%\item Barabanova O.V., Budaev M.E., Debolskiy A.V., Glebova E.S., Kukanova E.A., Melnik K.O., Platonov V.S., Sergeev D.E., Varentsov M.V., {\bf Zamyatina M.Yu.}, Zhelesnova I.V. \emph{The dynamics of the atmospheric boundary layer and its interaction with the underlying surface in the coastal zone of the White Sea}. \href{http://lomonosov-msu.ru/archive/Lomonosov_2011/structure_5_1476.htm}{Proceedings of the International Youth Science Forum Lomonosov-2011}, Moscow, Russia, 2011 (in Russian).
%\end{bibsection}
\end{innerlist}

\section{Awards}
\begin{innerlist}
\item[2015-2019] \textbf{Lord Zuckerman Studentship} \\
School of Environmental Sciences, University of East Anglia $|$ Norwich, UK
\item[2014-2015] \textbf{\href{https://www.uea.ac.uk/study/postgraduate/scholarships/the-simon-wharmby-postgraduate-scholarship-in-environmental-science}{Simon Wharmby Postgraduate Scholarship}} \\
School of Environmental Sciences, University of East Anglia $|$ Norwich, UK
\item[2012] \textbf{World Meteorological Organization travel grant}
\end{innerlist}

\section{Skills}
\begin{minipage}[t]{.6\textwidth}
  \textbf{Computing}
  \begin{innerlist}
  \item Operating system: Linux, Windows
  %\item Meteorological application suite (\href{http://mapmakers.ru/en/main/products/gis.aspx}{GIS Meteo},
%\href{http://www.scanex.ru/en/software/default.asp?submenu=meteogamma&id=index}{MeteoGamma})
  \item Data analysis: Python, \Matlab, R, NCL
  \item Version control: Git
  \item Document preparation: \LaTeX
  \item Familiar with ArcGIS
  \end{innerlist}
\end{minipage}%
\begin{minipage}[t]{.4\textwidth}
  \textbf{Languages}
  \begin{innerlist}
  \item Russian: native speaker
  \item English: fluent
  \item French: beginner
  \end{innerlist}
\end{minipage}

\section{Vocational $\newline$ Training}
\begin{innerlist}
\item[21-22 Apr 2016] \textbf{\href{https://www.metoffice.gov.uk/research/modelling-systems/dispersion-model}{NAME} workshop}
\begin{innerlist}
\item\small introduction to the Met Office Numerical Atmospheric dispersion Modelling Environment (NAME), installing and running the model locally or on \href{http://www.jasmin.ac.uk/}{JASMIN} super-computer
\end{innerlist}
\item[4-8 Jan 2016] \textbf{UKCA Theory and Practice Workshop}
\begin{innerlist}
\item\small introduction to the main components of the Met Office United Kingdom Chemistry and Aerosols (UKCA) model, including exercises on adaptation of the model to various research purposes, running it and solving common problems
\end{innerlist}
\item[16-18 Dec 2015] \textbf{Introduction to Unified Model}
\begin{innerlist}
\item\small introduction to the Met Office Unified Model (UM), providing theoretical and practical experience of setting up experiments and running the model
\end{innerlist}
\item[23-27 Nov 2015] \textbf{\href{https://www.ncas.ac.uk/index.php/en/introduction-to-atmospheric-science-course}{Introduction to Atmospheric Science}}
\begin{innerlist}
\item\small introduction to key concepts in atmospheric science, covering such topics as atmospheric dynamics, composition of the atmosphere and climate change
\end{innerlist}
\item[2015-2019] \textbf{\href{http://www.enveast.ac.uk/programme}{EnvEast Doctoral Training Programme}}
%\item[autumn 2011] \textbf{\href{http://gcc.aos.ecu.edu/}{Global Climate Change course}}
%\begin{innerlist}
%\item\small course taught by the East Carolina University (USA) in partnership with the Shandong University (China), Faculdade de Jaguariuna (Brazil), Lomonosov Moscow State University (Russia) and TUD SUD America de Mexico (Mexico)
%\end{innerlist}
\end{innerlist}

%\vspace{.1in}
%Completed on-line \textbf{\href{https://www.codecademy.com/}{Codecademy}} courses:
%\begin{itemize}
%		\item Python
%\end{itemize}
%
%Completed on-line \textbf{\href{https://www.coursera.org/}{Coursera}} courses:
%\begin{innerlist}
%        \item The Data Scientist’s Toolbox
%        \item \href{https://www.coursera.org/account/accomplishments/records/nuPFsbVDd55fwCAZ}{R Programming}
%        \item Getting and Cleaning Data
%        \item Exploratory Data Analysis
%\end{innerlist}
%
%\vspace{.1in}
%Completed on-line \textbf{\href{https://www.meted.ucar.edu/}{MetEd}} modules:
%\begin{innerlist}
%        \item Climate Change: Fitting the Pieces Together
%        \item Introduction to Climate Models
%        \item Introduction to Statistics in Climatology
%\end{innerlist}
%
%\vspace{.1in}
%\textbf{Member of the Royal Meteorological Society}

\section{Vocational $\newline$ Experience}
\begin{innerlist}
\item[Aug-Sep 2013] \textbf{Weather Forecaster} \\
Forecast and Briefing Service, Branch of Main Aviation Meteorological Centre, Sheremetyevo International Airport $|$ Moscow, Russia
\item[Jun-Jul 2013] \textbf{Technician} \\
Department of Actinometry, \href{http://www.momsu.ru/english.html}{Meteorological Observatory}, Lomonosov Moscow State University $|$ Moscow, Russia 
\end{innerlist}

\section{Interships}
\begin{innerlist}
\item[Jun-Jul 2013] Department of Actinometry, Meteorological Observatory, Lomonosov Moscow State University $|$ Moscow, Russia \\
Supervisor: Senior Research Scientist Olga A. Shilovtseva
\begin{innerlist}
\item\small indoor solar radiation measurements
\end{innerlist}
\item[Jun-Jul 2012] Laboratory of Program Maintenance and Information Support, Department of Environmental Pollution Assessments, \href{http://www.igce.ru/}{Institute of Global Climate and Ecology}, Roshydromet and Russian Academy of Sciences $|$ Moscow, Russia \\
Supervisor: Head of laboratory \href{http://www.igce.ru/page/gromov}{Sergey A. Gromov}
\begin{innerlist}
\item\small analysis of the \href{http://www.eanet.asia/}{Acid Deposition Monitoring Network in East Asia (EANET)} data
\end{innerlist}
\end{innerlist}

\section{Fieldwork}
\begin{innerlist}
\item[Jul-Aug 2012] \href{http://kronoki.org/}{Kronotsky Nature Reserve} $|$ Kamchatka peninsula, Russia
\begin{innerlist}
\item\small understanding prevailing mesoscale processes through wind measurements and lake hydrothermodynamical modelling
\end{innerlist}
\item[Jan-Feb 2011/12] \href{http://en.wsbs-msu.ru/}{White Sea Biological Station} $|$ Republic of Karelia, Russia
\begin{innerlist}
\item\small practical experience in using meteorological instruments in severe winter conditions
\item\small comparing observational data with analytical methods of calculating the dynamics of convective boundary layer over a polynya
\end{innerlist}
\item[Jun-Jul 2011] \href{http://www.eng.geogr.msu.ru/practics/stations/khibiny/}{Khibiny Teaching and Research Station} $|$ Murmansk Oblast, Russia
\begin{innerlist}
\item\small profiling of the atmosphere, measurements of atmospheric stability and solar radiation
%curricular practical training in atmospheric science
\end{innerlist}
\item[Jun-Jul 2010] \href{http://www.eng.geogr.msu.ru/practics/stations/satin/}{Satino Teaching and Research Station} $|$ Kaluga Oblast, Russia
\begin{innerlist}
\item\small curricular training in cartography, geology, geomorphology, soil science, biogeography, meteorology, hydrology and landscape science
\end{innerlist}
\end{innerlist}

\section{Teaching}
\begin{innerlist}
\item[Jan 2018] \textbf{Instructor} \\
Training course Introduction to Python in Environmental Sciences \\
University of East Anglia $|$ Norwich, UK
\item[2015-2018] \textbf{Associate Tutor} \\
Modules: Numerical Skills for Scientists, Atmospheric Chemistry and Global Change, Atmospheric Composition (Measurements and Modelling), Atmosphere $\&$ Oceans I \\
University of East Anglia $|$ Norwich, UK
\end{innerlist}

\section{Outreach}
\begin{innerlist}
\item[Nov 2015-present] Maintainer of {\href {https://twitter.com/atmoschemuea}{@AtmosChemUEA}} Twitter account
\item[Sep 2017-Jun 2018] Co-organiser of the \href{https://www.uea.ac.uk/environmental-sciences/news-and-events/atmospheric-and-marine-biogeochemistry-seminars}{Atmospheric and Marine Biogeochemistry seminars} \\
University of East Anglia $|$ Norwich, UK
\end{innerlist}

%\section{Volunteering}
%\vspace{.2in}
%\begin{innerlist}
%\item[Jan-Feb 2014] \textbf{Sochi2014 Olympic Games} $|$ Sochi, Russia \\
%Meteo-Office Assistant at the Sochi2014 Headquarters (Imeretinka)
%\item[Sep 2013] \textbf{London Wildlife Workcamp} (VAP UK-04) $|$ London, UK
%\begin{innerlist}
%%\item\small learned about wildlife in London with \href{http://www.vap.org.uk/}{Voluntary Action for Peace (VAP)} \
%\item\small conservation work with \href{http://www.wildlondon.org.uk/}{London Wildlife Trust} in nature reserves in South London \\
%\end{innerlist}
%\item[Mar 2013] \textbf{IPC Biathlon and Cross-Country Skiing World Cup finals} $|$ Sochi, Russia \\
%Meteo-Office Assistant at the Laura Biathlon and Ski Complex
%\item[Jul-Aug 2011] \textbf{Blossoming meadows} (SDA402) $|$ Vlasim, Czech Republic \\
%\begin{innerlist}
%\item\small took care of protected meadows with endangered species under the direction of \href{http://www.csop.cz/index.php?cis_menu=3&m1_id=1257}{Czech Union for Nature Conservation}
%\end{innerlist}
%\end{innerlist}

%\section{Other Work Experience}
%\textbf{Shop Assistant} \hfill {Jun 2012}
%\begin{innerlist}
%\item[] Stradivarius (retail, international women clothing fashion brand)
%\end{innerlist}

%\halfblankline

\end{document}
