%%%%%%%%%%%%%%%%%%%%%%%%%%%%%%%%%%%%%%%%%%%%%%%%%%%%%%%%%%%%%%%%%%%%%%%
%% NOTE: If you find that it says                                     %%
%%                                                                    %%
%%                           1 of ??                                  %%
%%                                                                    %%
%% at the bottom of your first page, this means that the AUX file     %%
%% was not available when you ran LaTeX on this source. Simply RERUN  %%
%% LaTeX to get the ``??'' replaced with the number of the last page  %%
%% of the document. The AUX file will be generated on the first run   %%
%% of LaTeX and used on the second run to fill in all of the          %%
%% references.                                                        %%
%%%%%%%%%%%%%%%%%%%%%%%%%%%%%%%%%%%%%%%%%%%%%%%%%%%%%%%%%%%%%%%%%%%%%%%%

\documentclass[10pt]{article}

% The automated optical recognition software used to digitize resume
% information works best with fonts that do not have serifs. This
% command uses a sans serif font throughout. Uncomment both lines (or at
% least the second) to restore a Roman font (i.e., a font with serifs).
%\usepackage{times}
%\renewcommand{\familydefault}{\sfdefault}

% This is a helpful package that puts math inside length specifications
\usepackage{calc}
\usepackage{comment}
\usepackage{amssymb}
\usepackage{MnSymbol}

% Simpler bibsection for CV sections
% (thanks to natbib for inspiration)
\makeatletter
\newlength{\bibhang}
\setlength{\bibhang}{1em} %1em}
\newlength{\bibsep}
 {\@listi \global\bibsep\itemsep \global\advance\bibsep by\parsep}
\newenvironment{bibsection}%
        {\begin{enumerate}{}{%
%        {\begin{list}{}{%
       \setlength{\leftmargin}{\bibhang}%
       \setlength{\itemindent}{-\leftmargin}%
       \setlength{\itemsep}{\bibsep}%
       \setlength{\parsep}{\z@}%
        \setlength{\partopsep}{0pt}%
        \setlength{\topsep}{0pt}}}
        {\end{enumerate}\vspace{-.6\baselineskip}}
%        {\end{list}\vspace{-.6\baselineskip}}
\makeatother

% Layout: Puts the section titles on left side of page
\reversemarginpar

%
%         PAPER SIZE, PAGE NUMBER, AND DOCUMENT LAYOUT NOTES:
%
% The next \usepackage line changes the layout for CV style section
% headings as marginal notes. It also sets up the paper size as either
% letter or A4. By default, letter was used. If A4 paper is desired,
% comment out the letterpaper lines and uncomment the a4paper lines.
%
% As you can see, the margin widths and section title widths can be
% easily adjusted.
%
% ALSO: Notice that the includefoot option can be commented OUT in order
% to put the PAGE NUMBER *IN* the bottom margin. This will make the
% effective text area larger.
%
% IF YOU WISH TO REMOVE THE ``of LASTPAGE'' next to each page number,
% see the note about the +LP and -LP lines below. Comment out the +LP
% and uncomment the -LP.
%
% IF YOU WISH TO REMOVE PAGE NUMBERS, be sure that the includefoot line
% is uncommented and ALSO uncomment the \pagestyle{empty} a few lines
% below.
%

%% Use these lines for letter-sized paper
%\usepackage[paper=letterpaper,
%            %includefoot, % Uncomment to put page number above margin
%            marginparwidth=1.2in,     % Length of section titles
%            marginparsep=.05in,       % Space between titles and text
%            margin=1in,               % 1 inch margins
%            includemp]{geometry}

%% Use these lines for A4-sized paper
\usepackage[paper=a4paper,
            %includefoot, % Uncomment to put page number above margin
            marginparwidth=30.5mm,    % Length of section titles
            marginparsep=1.5mm,       % Space between titles and text
            margin=25mm,              % 25mm margins
            includemp]{geometry}

%% More layout: Get rid of indenting throughout entire document
\setlength{\parindent}{0in}

\usepackage[shortlabels]{enumitem}

%% Reference the last page in the page number
%
% NOTE: comment the +LP line and uncomment the -LP line to have page
%       numbers without the ``of ##'' last page reference)
%
% NOTE: uncomment the \pagestyle{empty} line to get rid of all page
%       numbers (make sure includefoot is commented out above)
%
\usepackage{fancyhdr,lastpage}
\pagestyle{fancy}
%\pagestyle{empty}      % Uncomment this to get rid of page numbers
\fancyhf{}\renewcommand{\headrulewidth}{0pt}
\fancyfootoffset{\marginparsep+\marginparwidth}
\newlength{\footpageshift}
\setlength{\footpageshift}
          {0.5\textwidth+0.5\marginparsep+0.5\marginparwidth-2in}
\lfoot{\hspace{\footpageshift}%
       \parbox{4in}{\, \hfill %
                    \arabic{page} of \protect\pageref*{LastPage} % +LP
%                    \arabic{page}                               % -LP
                    \hfill \,}}

% Finally, give us PDF bookmarks
\usepackage{color,hyperref}
\definecolor{darkblue}{rgb}{0.0,0.0,0.25}
\hypersetup{colorlinks,breaklinks,
            linkcolor=darkblue,urlcolor=darkblue,
            anchorcolor=darkblue,citecolor=darkblue}

\usepackage{graphicx}
\usepackage{wrapfig}
%%%%%%%%%%%%%%%%%%%%%%%% End Document Setup %%%%%%%%%%%%%%%%%%%%%%%%%%%%


%%%%%%%%%%%%%%%%%%%%%%%%%%% Helper Commands %%%%%%%%%%%%%%%%%%%%%%%%%%%%

% The title (name) with a horizontal rule under it
% (optional argument typesets an object right-justified across from name
%  as well)
%
% Usage: \makeheading{name}
%        OR
%        \makeheading[right_object]{name}
%
% Place at top of document. It should be the first thing.
% If ``right_object'' is provided in the square-braced optional
% argument, it will be right justified on the same line as ``name'' at
% the top of the CV. For example:
%
%       \makeheading[\emph{Curriculum vitae}]{Your Name}
%
% will put an emphasized ``Curriculum vitae'' at the top of the document
% as a title. Likewise, a picture could be included:
%
%   \makeheading[\includegraphics[height=1.5in]{my_picutre}]{Your Name}
%
% the picture will be flush right across from the name.
\newcommand{\makeheading}[2][]%
        {\hspace*{-\marginparsep minus \marginparwidth}%
         \begin{minipage}[t]{\textwidth+\marginparwidth+\marginparsep}%
             {\large \bfseries #2 \hfill #1}\\[-0.15\baselineskip]%
                 \rule{\columnwidth}{1pt}%
         \end{minipage}}

% The section headings
%
% Usage: \section{section name}
\renewcommand{\section}[1]{\pagebreak[3]%
    \hyphenpenalty=10000%
    \vspace{1.3\baselineskip}%
    \phantomsection\addcontentsline{toc}{section}{#1}%
    \noindent\llap{\scshape\smash{\parbox[t]{\marginparwidth}{\raggedright #1}}}%
    \vspace{-\baselineskip}\par}

% An itemize-style list with lots of space between items
\newenvironment{outerlist}[1][\enskip\textbullet]%
        {\begin{itemize}[#1,rightmargin=2mm,leftmargin=*]}{\end{itemize}%
         \vspace{-.6\baselineskip}}

% An environment IDENTICAL to outerlist that has better pre-list spacing
% when used as the first thing in a \section
\newenvironment{lonelist}[1][\enskip\textbullet]%
        {\begin{list}{#1}{%
        \setlength{\partopsep}{0pt}%
        \setlength{\topsep}{0pt}}}
        {\end{list}\vspace{-.6\baselineskip}}

% An itemize-style list with little space between items
\newenvironment{innerlist}[1][\enskip$\filledsquare$]%
        {\begin{itemize}[#1,leftmargin=*,rightmargin=1mm,parsep=0pt,itemsep=0pt,topsep=0pt,partopsep=0pt]}
        {\end{itemize}}
        
\newenvironment{list2}[1][\enskip$\filledsquare$]%
        {\begin{itemize}[#1,leftmargin=*,rightmargin=4cm,parsep=0pt,itemsep=0pt,topsep=0pt,partopsep=0pt]}
        {\end{itemize}}

% An environment IDENTICAL to innerlist that has better pre-list spacing
% when used as the first thing in a \section
\newenvironment{loneinnerlist}[1][\enskip\textbullet]%
        {\begin{itemize}[#1,leftmargin=*,parsep=0pt,itemsep=0pt,topsep=0pt,partopsep=0pt]}
        {\end{itemize}\vspace{-.6\baselineskip}}

% To add some paragraph space between lines.
% This also tells LaTeX to preferably break a page on one of these gaps
% if there is a needed pagebreak nearby.
\newcommand{\blankline}{\quad\pagebreak[3]}
\newcommand{\halfblankline}{\quad\vspace{-0.5\baselineskip}\pagebreak[3]}

% Uses hyperref to link DOI
\newcommand\doilink[1]{\href{http://dx.doi.org/#1}{#1}}
\newcommand\doi[1]{doi:\doilink{#1}}

% For \url{SOME_URL}, links SOME_URL to the url SOME_URL
\providecommand*\url[1]{\href{#1}{#1}}
% Same as above, but pretty-prints SOME_URL in teletype fixed-width font
\renewcommand*\url[1]{\href{#1}{\texttt{#1}}}

% For \email{ADDRESS}, links ADDRESS to the url mailto:ADDRESS
\providecommand*\email[1]{\href{mailto:#1}{#1}}
% Same as above, but pretty-prints ADDRESS in teletype fixed-width font
%\renewcommand*\email[1]{\href{mailto:#1}{\texttt{#1}}}

%\providecommand\BibTeX{{\rm B\kern-.05em{\sc i\kern-.025em b}\kern-.08em
%    T\kern-.1667em\lower.7ex\hbox{E}\kern-.125emX}}
%\providecommand\BibTeX{{\rm B\kern-.05em{\sc i\kern-.025em b}\kern-.08em
%    \TeX}}
\providecommand\BibTeX{{B\kern-.05em{\sc i\kern-.025em b}\kern-.08em
    \TeX}}
\providecommand\Matlab{\textsc{Matlab}}

%%%%%%%%%%%%%%%%%%%%%%%% End Helper Commands %%%%%%%%%%%%%%%%%%%%%%%%%%%

%%%%%%%%%%%%%%%%%%%%%%%%% Begin CV Document %%%%%%%%%%%%%%%%%%%%%%%%%%%%

\begin{document}
\makeheading[\emph{Curriculum vitae}]{Maria Zamyatina}

\begin{wrapfigure}{r}{0.25\textwidth}
\includegraphics[width=0.25\textwidth]{photoCV_whitebackground.jpg}
\end{wrapfigure}

\section{Personal Information}
% NOTE: Mind where the & separators and \\ breaks are in the following
%       table.
%
% ALSO: \rcollength is the width of the right column of the table
%       (adjust it to your liking; default is 1.85in).
%
%\newlength{\rcollength}\setlength{\rcollength}{1.4in}%
%
%\begin{tabular}[t]{@{}p{\textwidth-\rcollength}p{\rcollength}}
University of East Anglia, Norwich Research Park, Norwich, NR4 7TJ, UK\\
+44 7925606193 \\
\email{m.zamyatina@uea.ac.uk}
%\end{tabular}%

\section{Research Interests}

\begin{innerlist}
\item  Atmospheric chemistry 
\item  Climate change
\item  Numerical modelling
\item  Paleoclimatology
\item  Synoptic meteorology
\end{innerlist}

\section{Education}
\textbf{PhD in Environmental Sciences}  \hfill {Oct 2015 - present} \\
\href{https://www.uea.ac.uk/environmental-sciences/}{School of Environmental Sciences} \\
\href{https://www.uea.ac.uk/}{University of East Anglia}
\begin{innerlist}
\item Thesis title: \emph{Is human activity affecting the atmosphere’s
ability to clean itself of pollutants?}
\item Supervisor: \href{mailto:c.reeves@uea.ac.uk}{Prof. Claire Reeves}
\end{innerlist}

\vspace{.1in}
\textbf{MSc in Climate Change with Distinction}  \hfill {2015} \\
\href{https://www.uea.ac.uk/environmental-sciences/}{School of Environmental Sciences} \\
\href{https://www.uea.ac.uk/}{University of East Anglia}
\begin{innerlist}
\item Thesis title: \emph{Investigation of the relationship between tropospheric ozone production efficiency and carbon bond emissions}
\item Supervisor: \href{mailto:c.reeves@uea.ac.uk}{Prof. Claire Reeves}
\end{innerlist}

\vspace{.1in}
\textbf{Specialist Diploma in Meteorology}  \hfill {2014} \\
\href{http://www.eng.geogr.msu.ru/}{Faculty of Geography} \\
\href{http://www.msu.ru/en/}{Lomonosov Moscow State University}
\begin{innerlist}
\item Thesis title: \emph{Climatically-induced variations of the Caspian Sea level over the last Millennium}
\item Supervisor: \href{mailto:avkislov@mail.ru}{Prof. Alexander V. Kislov}
\end{innerlist}

\section{Research Experience}
\textbf{Internship in Actinometry} \hfill {Jun-Jul 2013}
\begin{innerlist}
\item[] Department of Actinometry, \\
        \href{http://www.momsu.ru/english.html}{Meteorological Observatory}, \\
        Lomonosov Moscow State University \\
        Supervisor: Senior Research Scientist, Olga A. Shilovtseva
\end{innerlist}
\vspace{.1in}
        
\textbf{Internship in Atmospheric Chemistry} \hfill {Jun-Jul 2012}
\begin{innerlist}
\item[] Laboratory of Program Maintenance and Information Support, \\
        Department of Environmental Pollution Assessments, \\
        \href{http://www.igce.ru/}{Institute of Global Climate and Ecology}, \\
        Roshydromet and Russian Academy of Sciences\\
        Supervisor: \href{mailto:Sergey.Gromov@igce.ru}{Head of laboratory, Sergey A. Gromov} \
\item\small data processing of regional and background environmental contamination in 
            East Asia based on data of \href{http://www.eanet.asia/}{Acid Deposition Monitoring Network
            in East Asia (EANET)}
\item\small meteorological and climatological analysis of observational data
\end{innerlist}

\section{Professional Experience}
\textbf{Weather Forecaster} \hfill {Aug-Sep 2013}
\begin{innerlist}
\item[] Forecast and Briefing Service, \\
        Branch of Main Aviation Meteorological Centre, \\
        Sheremetyevo International Airport, Moscow \\
\end{innerlist}

\textbf{Technician} \hfill {Jun-Jul 2013}
\begin{innerlist}
\item[] Department of Actinometry, \\
        \href{http://www.momsu.ru/english.html}{Meteorological Observatory}, \\
        Lomonosov Moscow State University   
\end{innerlist}

\section{Fieldwork Experience}

\textbf{Field research in Meteorology} \hfill {Jul-Aug 2012}
\begin{innerlist}
\item[] \href{http://kronoki.org/}{Kronotsky Nature Reserve}, Kamchatka pen., Russia \
\item\small attempt to understanding of prevailing mesoscale processes through wind characteristic
            measurements and lake hydrothermodynamical modelling
\end{innerlist}

\vspace{.1in}
\textbf{Field research in Meteorology} \hfill {Jan-Feb 2012}
\begin{innerlist}
\item[] \href{http://en.wsbs-msu.ru/}{White Sea Biological Station}, Republic of Karelia, Russia \
\item\small comparison of analytical methods for calculating the dynamics of convective boundary
            layer over polynya with observational data
\end{innerlist}

\vspace{.1in}
\textbf{Practical training in Meteorology} \hfill {Jun-Jul 2011}
\begin{innerlist}
\item[] \href{http://www.eng.geogr.msu.ru/practics/stations/khibiny/}{Khibiny Teaching and Research Station}, Murmansk Oblast, Russia \
\item\small curricular practical trainings in atmospheric science basic field techniques (profiling of atmosphere, atmospheric stability and radiative measurements)
\end{innerlist}

\vspace{.1in}
\vspace{.1in}
\textbf{Field research in Meteorology} \hfill {Jan-Feb 2011}
\begin{innerlist}
\item[] \href{http://en.wsbs-msu.ru/}{White Sea Biological Station}, Republic of Karelia, Russia
\item\small gained practical experience of using various meteorological instruments in severe winter conditions
\end{innerlist}

\vspace{.1in}
\textbf{General geographic practical training} \hfill {Jun-Jul 2010}
\begin{innerlist}
\item[] \href{http://www.eng.geogr.msu.ru/practics/stations/satin/}{Satino Teaching and Research Station}, Kaluga Oblast, Russia \
\item\small curricular practical trainings in topography, geology and geomorphology, soil science, biogeography, meteorology, hydrology and landscape science
\end{innerlist}

\section{Teaching experience}
\textbf{Associate Tutor} \hfill {Oct 2015 - present}
\begin{innerlist}
\item[] University of East Anglia \\
        Modules:
        \begin{innerlist}
        \item\small Numerical Skills for Scientists
        \item\small Atmospheric Chemistry and Global Change 
        \end{innerlist}
\end{innerlist}

\section{Publications}
\begin{bibsection}
\item Gromov S.A., Gromov S.S., {\bf Zamyatina M.Yu.}, Trifonova-Yakovleva A.M. \emph{First-order evaluation of transboundary pollution fluxes in areas of EANET stations in Eastern Siberia and the Russian Far East}. \href{http://www.eanet.asia/product/science_bulletin/Sep_2013.pdf}{EANET Science Bulletin Vol. 3, p. 195-203, 2013}.
\item Sergeev D., {\bf Zamyatina M.}, Stepanenko V. \emph{Features of temperature regime of Kronotskoe lake (Kronotsky Nature Reserve)}. \href{http://istina.msu.ru/publications/article/5336939/}{Proceedings of Kronotsky Nature Reserve Vol. 3. p. 29-41, 2013} (in Russian).
\end{bibsection}

\section{Conference Proceedings}
\begin{innerlist}
\item Gromov S.A., Gromov S.S., {\bf Zamyatina M.Yu.} 2012. \emph{Assessment of climatological potential of trasboudary air pollution transport in Eastern Siberia and the Russian Far East}. Proceedings of the \href{http://aqm2012.itu.edu.tr/}{Air Quality Management at Urban, Regional and Global Scales 4th International Symposium and IUAPPA Regional Conference (AQM2012)}, Istanbul, Turkey. Selected papers: 109.
\item Barabanova O.V., Fedorov G.A., Khrupolova E.A., Konstantinov P.I., Kukanova E.A., Malinina E.P., Sergeev D.E., Sokolova L.A., Stepanenko V.M., Varentsov M.V., Veresemskaya P.S., {\bf Zamyatina M.Yu.}, Zheleznova I.V. 2012. \emph {Experimental investigation and remote sensing of boundary layer in high latitudes (evidence from the coastal zone of the White Sea)}. \href{http://lomonosov-msu.ru/archive/Lomonosov_2012/structure_6_1716.htm}{Proceedings of the International Youth Science Forum Lomonosov-2012}, Moscow, Russia (in Russian).
\item Barabanova O.V., Budaev M.E., Debolskiy A.V., Glebova E.S., Kukanova E.A., Melnik K.O., Platonov V.S., Sergeev D.E., Varentsov M.V., {\bf Zamyatina M.Yu.}, Zhelesnova I.V. 2011. \emph{The dynamics of the atmospheric boundary layer and its interaction with the underlying surface in the coastal zone of the White Sea}. \href{http://lomonosov-msu.ru/archive/Lomonosov_2011/structure_5_1476.htm}{Proceedings of the International Youth Science Forum Lomonosov-2011}, Moscow, Russia (in Russian).
\end{innerlist}

\section{Poster presentations}
\begin{innerlist}
\item {\bf Zamyatina M.}, Reeves C.E., Newland M.J., Archibald A.T., Griffiths P.T. \emph{Is human activity affecting the atmosphere’s ability to clean itself of pollutants?}. UKCA Theory and Practice Workshop, 4-8 January 2016, Cambridge, UK.
\end{innerlist}

\section{Scholarships and Awards}
\textbf{Lord Zuckerman Studentship} \hfill {2015}
\begin{innerlist}
\item[] School of Environmental Sciences, University of East Anglia
\end{innerlist}
\vspace{.1in}
\textbf{Simon Wharmby Postgraduate Scholarship} \hfill {2014/15}
\begin{innerlist}
\item[] School of Environmental Sciences, University of East Anglia
\end{innerlist}
\vspace{.1in}
\textbf{World Meteorological Organization travel financial support} \hfill {2012}

\section{Skills}
\textbf{Languages}
\begin{innerlist}
\item Russian (mother-tongue)
\item English (IELTS 7.0)
%\item French (reading knowledge)
\end{innerlist}

\vspace{.1in}
\textbf{Computing Skills}
\begin{innerlist}
%\item Operating system (Windows)
\item Meteorological application suite (\href{http://mapmakers.ru/en/main/products/gis.aspx}{GIS Meteo},
\href{http://www.scanex.ru/en/software/default.asp?submenu=meteogamma&id=index}{MeteoGamma})
\item Data analysis and visualization (\Matlab, Python, R, NCL, NCO)
\item Version control systems (Git)
\item Microsoft Office (Word, Excel, PowerPoint)
\item Familiar with ArcGIS, \LaTeX
\end{innerlist}

\section{Personal and Professional Development}
\textbf{UKCA Theory and Practice Workshop} \hfill {4-8 Jan 2016} \\
Rewrite!\\
This course is designed to give new users a good understanding of the theory behind the UKCA model code, such as aerosol microphysics, dry and wet deposition, and the chemical mechanisms used, as well as teaching how to use the model.
After attending this course, I feel more knowledgeable about the main components of the UKCA model and more confident in adapting it for my own purposes. Moreover, since this course builds on the experience gained during the 'Introduction to Unified Model' (UM) course (which I attended previously), I used this opportunity to revise various aspects of UM that are directly related to UKCA. These aspects are: 
adding new species and reactions,
changing and adding emissions,
managing output, and
adding new diagnostics.

\vspace{.1in}
\textbf{Introduction to Unified Model} \hfill {16-18 Dec 2015} \\
Rewrite!\\
Introduction to The Met Office Unified Model, including the set-up interface, how
to run the model and the outline of research that is being carried out using the
model\\
This course introduces new users to the Met Office Unified Model (UM) systems and provides practical experience of setting up and running experiments. UM software management system, file formats, utilities, and configurations (FCM) were discussed. I attended all lectures and completed a series of exercises designed to explore the UMUI and ROSE user interface functionality, ran UM on ARCHER supercomputer and learnt how to solve common UM problems. I also did further exercises on post-processing of output data from experiments, where we were asked to change, e.g. auxiliary files (STASH), or use extra scripts and hand edits.

\vspace{.1in}
\textbf{Introduction to Atmospheric Science} \hfill {23-27 Nov 2015} \\
Rewrite!\\
This course is designed to introduce key concepts in atmospheric science to 1st and 2nd year PhD students. I attended all lectures and did all short exercises given during this course. The lectures coved such topics as climate, composition of the atmosphere and weather atmospheric dynamics. I also took part in public engagement activities, namely helped my group to prepare a tweet, a press release and a public lecture abstract aimed to communicate results of a given journal publication.

\vspace{.1in}
\textbf{\href{http://gcc.aos.ecu.edu/}{Global Climate Change course}} \hfill {2011(fall)} \\
Successfully completed Global Climate Change course taught by East Carolina University (USA) in partnership with Shandong University (China), Faculdade de Jaguariuna (Brazil), Lomonosov Moscow State University (Russia) and TUD SUD America de Mexico (Mexico)

\vspace{.1in}
Completed on-line \textbf{\href{https://www.codecademy.com/}{Codecademy}} courses:
\begin{itemize}
		\item Python
\end{itemize}

Completed on-line \textbf{\href{https://www.coursera.org/}{Coursera}} courses:
\begin{innerlist}
        \item The Data Scientist’s Toolbox
        \item \href{https://www.coursera.org/account/accomplishments/records/nuPFsbVDd55fwCAZ}{R Programming}
        \item Getting and Cleaning Data
        \item Exploratory Data Analysis
\end{innerlist}

\vspace{.1in}
Completed on-line \textbf{\href{https://www.meted.ucar.edu/}{MetEd}} modules:
\begin{innerlist}
        \item Climate Change: Fitting the Pieces Together
        \item Introduction to Climate Models
        \item Introduction to Statistics in Climatology
\end{innerlist}

\vspace{.1in}
\textbf{Member of the Royal Meteorological Society}

\section{Outreach}
Co-maintainer of the AtmosChemUEA twitter account.

\section{Voluntary Work Experience}
\textbf{Sochi2014 Olympic Games} \hfill {Jan-Feb 2014}
\begin{innerlist}
\item[] Sochi, Russia \\
        {\em Meteo-office Assistant} at Sochi2014 Headquarters (Imeretinka) \\   
\end{innerlist}

\textbf{London Wildlife Workcamp} \hfill {Sep 2013}
\begin{innerlist}
\item[] London, United Kingdom \\
        VAP UK-04       
\item\small learned about wildlife in London with \href{http://www.vap.org.uk/}{Voluntary Action for Peace (VAP)} \
\item\small practical conservation work with \href{http://www.wildlondon.org.uk/}{London Wildlife Trust} on some of their more remote nature reserves in South London \\
\end{innerlist}

\textbf{IPC Biathlon and Cross-Country Skiing World Cup finals} \hfill {Mar 2013}
\begin{innerlist}
\item[] Sochi, Russia \\
        {\em Meteo-office Assistant} at the Laura Biathlon and Ski Complex \\
\end{innerlist}

\textbf{Blossoming meadows} \hfill {Jul-Aug 2011}
\begin{innerlist}
\item[] Vlasim, Central Bohemian Region, Czech Republic \\
        SDA402
\item\small took care of protected meadows with endangered species under the direction of \href{http://www.csop.cz/index.php?cis_menu=3&m1_id=1257}{Czech Union for Nature Conservation}
\end{innerlist}

\section{Other Work Experience}
\textbf{Shop Assistant} \hfill {Jun 2012}
\begin{innerlist}
\item[] Stradivarius (retail, international women clothing fashion brand)
\end{innerlist}

%\halfblankline

\end{document}
